\part{State of the art}
\label{chap:general}

\section*{Phone}
\begin{itemize}
\item Téléphone portable technologie qu'on a pratiquement toujours sur soit et allumé
\item Il existe plusieurs façon de localiser un utilisateur de téléphone
\item Depuis le téléphone
  \begin{itemize}
  \item Triangulation via les antennes de GSM
  \item GPS présent dans quasi tous les smartphones, maintenant même dans les appareils photos
  \item Wifi (cf Google, voir partie android)
  \item Malware installé sur l'appareil, cas Flexispy ou Carrier IQ
  \end{itemize}
\item L'opérateur ou un extérieur
  \begin{itemize}
  \item Données RAW dans les trâmes captées par les opérateurs permettent de connaitre la puissance du signal reçu, antennes...
  \item SMS furtif envoyé par les autorités
  \item Norme 112/911 qui oblige à pourvoir localiser n'importe qui à n'importe quel moment avec une précision relative -> problème du mode d'appel d'urgence sans déverouiller la carte SIM
  \end{itemize}
\item Scandale avec les iPhones traceurs
  \begin{itemize}
  \item Présentation des découvertes
  \item Pas rentrer dans la technique, laissée pour partie 2 section \ref{sec:andro-cell-db}
  \end{itemize}
\end{itemize}

\section*{RFID}
\begin{itemize}
\item Présent dans plus en plus de produits
\item Parfois toujours actif quand plus utile (sortie de magasin)
\item Distance de lecture variable (\emph{cf recherches faites pour lecture à plus longue distance})
\item Si réseau comme carte d'accès UCL était corrompu, pourrait localiser qui va où et quand (\emph{pas fiable à 100\%, rentre à plusieurs en même temps, portes pas toujours fermées})
\end{itemize}

\section*{Wifi}
\begin{itemize}
\item Cas Wifi UCL
  \begin{itemize}
  \item Couvre toute la ville de LLN
  \item Avec smartphone se connecte d'en plus en plus (3G encore cher)
  \item Peut savoir où les étudiants se trouvent
  \item Exemple étudiant prétent ne pas venir à un TP car malade, prof peut vérifier si était connecté à un wifi quelque part en ville
  \end{itemize}
\item Cas FON
  \begin{itemize}
  \item Partout en Belgique depuis l'accord avec Belgacom
  \item Peut connaitre les déplacements dans les villes
  \end{itemize}

\end{itemize}

% \chapter{Other}
% \begin{itemize}
% \item Carte banquaires
% \item Bluetooth
% \end{itemize}