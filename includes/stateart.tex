\chapter{State of the art}
\label{chap:general}

\textbf{Unfinished chapter!}

\section*{Introduction}
\label{sec:soa-intro}

This thesis study the privacy aspect of the different common technology used nowadays.
The two selected technologies selected for the case study are Android smartphones in Part \ref{part:android} and personal surveillance camera in Part \ref{part:camera}.
However, several other technologies have interesting privacy aspects, especially in term of traceability.
These technologies are described in this chapter.
The goal is not to cover every technologies but to open a reflection about the traceability possibilities by describing general localisation techniques.

\section{Phone}

The mobile phone is probably the device than most of the people posses and carries with himself at all time.
In 2011, the global penetration of the mobile phone was about 70\%\footnote{Saylor, Michael (2012). The Mobile Wave: How Mobile Intelligence Will Change Everything. Perseus Books/Vanguard Press. p. 5}.
The level of technology included in a mobile phone varies greatly between devices but this section considers general mobile phone devices with a low level of software capabilities to the advanced smartphones.
There exists several methods to locate a mobile device.
It could be done from a device or by external third parties.

\subsection{Native localisation methods}

Depending of the level of sophistication in the software system, a mobile phone is capable of computing basic localisation information and compromise the privacy of a user.\\

A mobile phone can be localised using several methods depending of the phone resources.
It could use the unique identifier of GSM cell towers or wireless access points to establish the area where the device is located.
On more advanced phones, a GPS receiver is also present to locate a device with a greater accuracy if in adequate conditions.
These methods are often available on advanced devices and are discussed in Chapter \ref{chap:and-loc}.

% Each GSM cell tower has a unique identifier, using this identifier allows a device to locate itself in a range of about one kilometre.
% The accuracy of the localisation can be increased using more advanced triangulation algorithms.
% Section \ref{sec:loc-cell-tower} explains the localisation using GSM cell towers used in Android.\\

% The mobile phones having wifi connection capabilities can be located using this technique

% Advanced mobile phones posses also GPS capabilities.
% If in the adequate conditions, the GPS can retrieve a precise position of a mobile phone.
% This feature is however present mostly on smartphones and studied in more details in Section \ref{sec:loc-gps}.

\subsection{Flexispy}
\label{sec:flexispy}

FlexiSPY\footnote{\url{http://www.flexispy.com/}} is a mobile phone software available for Android, iOS, BlackBerry, Windows Mobile and Symbian (Nokia) systems.
This software is sold as a \emph{spyphone}, a way to discover partner's secrets and cheating wives.
It claims to have a full control on the content of the phone on which the software is installed from reading SMS to detecting SIM card change or intercept calls and localisation on a map.
This software is considered as a spyware by several antivirus such as Symantec or F-Secure even if sold as service.

\subsection{E911/E112}
\label{sec:soa-911}

The services E911, for \emph{Enhanced 9-1-1}, and E112 are location-enhanced versions of the emergency numbers 911 (US) and 112 (UE).
US and UE directives require mobile phone networks to provide emergency services with whatever information they have about the location a mobile call was made.
While non-mobile phones have an address associated to the line (which allows a fast location), the mobile phones are harder to locate for the network provider.


% \begin{itemize}
% \item Téléphone portable technologie qu'on a pratiquement toujours sur soit et allumé
% \item Il existe plusieurs façon de localiser un utilisateur de téléphone
% \item Depuis le téléphone
%   \begin{itemize}
%   \item Triangulation via les antennes de GSM
%   \item GPS présent dans quasi tous les smartphones, maintenant même dans les appareils photos
%   \item Wifi (cf Google, voir partie android)
%   \item Malware installé sur l'appareil, cas Flexispy ou Carrier IQ
%   \end{itemize}
% \item L'opérateur ou un extérieur
%   \begin{itemize}
%   \item Données RAW dans les trâmes captées par les opérateurs permettent de connaitre la puissance du signal reçu, antennes...
%   \item SMS furtif envoyé par les autorités
%   \item Norme 112/911 qui oblige à pourvoir localiser n'importe qui à n'importe quel moment avec une précision relative -> problème du mode d'appel d'urgence sans déverouiller la carte SIM
%   \end{itemize}
% \item Scandale avec les iPhones traceurs
%   \begin{itemize}
%   \item Présentation des découvertes
%   \item Pas rentrer dans la technique, laissée pour partie 2 section \ref{sec:andro-cell-db}
%   \end{itemize}
% \end{itemize}

\section{RFID}

\begin{itemize}
\item Présent dans plus en plus de produits
\item Parfois toujours actif quand plus utile (sortie de magasin)
\item Distance de lecture variable (\emph{cf recherches faites pour lecture à plus longue distance})
\item Si réseau comme carte d'accès UCL était corrompu, pourrait localiser qui va où et quand (\emph{pas fiable à 100\%, rentre à plusieurs en même temps, portes pas toujours fermées})
\end{itemize}

\section{Wifi}

\begin{itemize}
\item Cas Wifi UCL
  \begin{itemize}
  \item Couvre toute la ville de LLN
  \item Avec smartphone se connecte d'en plus en plus (3G encore cher)
  \item Peut savoir où les étudiants se trouvent
  \item Exemple étudiant prétent ne pas venir à un TP car malade, prof peut vérifier si était connecté à un wifi quelque part en ville
  \end{itemize}
\item Cas FON
  \begin{itemize}
  \item Partout en Belgique depuis l'accord avec Belgacom
  \item Peut connaitre les déplacements dans les villes
  \end{itemize}

\end{itemize}

% \chapter{Other}
% \begin{itemize}
% \item Carte banquaires
% \item Bluetooth
% \end{itemize}