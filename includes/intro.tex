\chapter{Thesis introduction}

\section{The context}

Extract from article ``The Really Smart Phone'' in the Wall Street journal (April 22, 2011)~\cite{really-smart-phone}.

\begin{quotation}
  [\dots]
  As a tool for field research, the cellphone is unique. Unlike a
  conventional land-line telephone, a mobile phone usually is used by
  only one person, and it stays with that person everywhere,
  throughout the day.
  [\dots]\\

  Advances in statistics, psychology and the
  science of social networks are giving researchers the tools to find
  patterns of human dynamics too subtle to detect by other means. At
  Northeastern University in Boston, network physicists discovered
  just how predictable people could be by studying the travel routines
  of 100,000 European mobile-phone users.
  [\dots]\\

  After analyzing more than
  16 million records of call date, time and position, the researchers
  determined that, taken together, people's movements appeared to
  follow a mathematical pattern. The scientists said that, with enough
  information about past movements, they could forecast someone's
  future whereabouts with 93.6\% accuracy.
  The pattern held true whether people stayed close to home or
  traveled widely, and wasn't affected by the phone user's age or
  gender.
  [\dots]\\
\end{quotation}


We introduce the context of the current thesis with this extract of an article published in 2011 in the Wall Street Journal.
This gives a rather frightening view on the potential from the exploitation of the massive amount of data generated by our smart phones.
One year later one can only confirm this structural trend.\\

The usage of technology is more and more embedded in our everyday life.
It keeps us organised and our ubiquitous interactions with others in a way that is less and less conscious.
Even more we get so used to the convenience of those technologies that we might get less and less reluctant at unveiling a part of our private life.
This is clearly a new way of life.\\

This thesis does not focus on computers or websites on the Internet but on the devices implementing this technology that contributes to our new way of life.
%These devices work as a black-box, if they \emph{just work} is usually enough for the usage we have of it.
Those devices are designed and marketed more as easy-to-use appliances than as computers with their complexity and known vulnerabilities.
Actually a modern smartphone is definitely not an appliance but a general purpose computer fitting in a very small and convenient case.
From the end-user perspective the privacy aspect of such a device is usually ignored or considered as good enough.
Also the end-user confidence in the manufacturer’s concern to deal with privacy is a recurrent problem in the privacy scandals\footnote{In April 2011, a cache file containing 8 months of previous user locations has been discovered on iPhones, Section \ref{sec:loc-cache-files} explains the purpose of such file and its presence on Android.}.

\section{Case study choices}

Although there are many very important and interesting aspects\footnote{Marketing, social and legal aspects to name some.} related to this new ``world'', this thesis will look at technical capabilities and related weaknesses.
As part of this thesis, it has been decided to focus on two types of devices.
\begin{itemize}
\item Smartphone is an obvious candidate for reasons mentioned above.
\item Wireless camera is an interesting candidate because much less mentioned in literature while privacy concerns are real.
\end{itemize}

The smartphones penetration rate is in constant raise and, in a few years, it is expected that the large majority of persons own such device in our countries.
Similarly to our personal computers, these devices contain more and more personal data.
The consequences of compromising these data can be very high for an end-user.
The cause of compromising can be both related to potential security weaknesses or to abuses from softwares for marketing purposes.\\

The wireless camera market is not as developed as for the smartphones but the privacy risks are acknowledged.
The consequences of camera compromising and unauthorised access to the video stream is potentially very high and can have the opposite effect than the one expected (to ease abuses instead of protecting).\\

For each device type, a concrete case has been selected for the sake of demonstrating
vulnerabilities and privacy consequences.
\begin{itemize}
\item The Android operating system has been selected for smartphones.
\item The D-Link DSC 2130 camera has been selected for wireless cameras.
\end{itemize}
These two choices are justified in the related parts.

\section{Potential abuses}

Beyond the two study cases, several technologies have potential abuses related to the privacy of the users.
This section does not intend to cover every technology but a brief presentation and opening a reflection on the risks of relying appliances for everyday tasks.\\

\subsection{Mobile phones}

As the Wall Street Journal extract introduced, the amount of data collected using mobile phones reveals patterns in the behaviour of human beings.
While this research focused on voluntary tracked family using smartphones, there exists many other unconscious way of tracking a mobile phone.\\

For instance, a phone operator is capable of locating any users in a range of a few hundreds of meters.
This is localisation is possible using the GSM cell towers information on which a user is connected.
This is a requirement imposed by US and European directives (E911 and E112) to be able to locate a mobile phone owner in case of emergency~\cite{e112-recom}.

\subsection{RFID}

RFID tags are present in more and more products, from clothes to credit card or passports, their uses varies from inventory tracking to identification or transactions.
As long as the distance from the reader is sufficient (from a few centimetres to meters depending on the tag), an RFID tag enables the transmission of information without any interaction needed from the person who carry the tag.
The communication can even happen without the user realising it.
Due to this last point, RFID technology is often involved in privacy concerns scandals.\\

RFID criticisms are based on the possibles traceability issues as well as the privacy concerns.
Privacy advocates ask for the destruction of RFID tags when not use (for instance after buying an item where the tag was used for inventory) or cover it with a aluminium foil.\\

RFID tags are now used in credit cards for contactless payments.
In January 2012, Kristin Paget demonstrated\footnote{Presentation of the hack was done during the Shmoocon 2012 conference, slides available at \url{http://www.tombom.co.uk/Paget-shmoocon-credit-cards.pdf}} that, with compatible RFID reader, it is possible to retrieve sensible bank information from the card such as credit card's number and expiration date, along with the one-time CVV number used by contactless cards to authenticate payments.

\subsection{Wifi hotspot}

More and more companies or internet providers offer an wifi sharing capabilities.
It is the case for instance with the international FON\footnote{FON network \url{http://corp.fon.com/}} system and, at a local scope, the wireless system deployed in the centre of Louvain-La-Neuve for the students and members of the UCL\footnote{Université Catholique de Louvain, Belgium \url{https://www.uclouvain.be/en-wifi.html}}.
These technologies are very convenient as allow to access the Internet outside the home network.\\

However these technologies introduce a security risk.
It is possible to create a fake access point on which victims may connect and reveals, in the best case, their credentials for the access point or more sensible information during the surf\footnote{During the master in computer sciences at the UCL, it has been demonstrated the feasibility of such attack. A fake FON access point was created to retrieve FON credentials as well as passwords during the surf using inexpensive material.}.

\section{Thesis document organisation}

The thesis is divided into two distinct parts.
The content of each part is independent and the order of reading does not affect the general comprehension.\\

\textbf{Part \ref{part:android}} is about the Android system and is composed of four chapters.\\

\textbf{Chapter \ref{chap:andro-intro}} presents and justifies the choice of studying the Android system as well as the focus on the localisation.
\textbf{Chapter \ref{chap:and-loc}} explains the localisation techniques used in Android as well as the privacy aspect related to it.
\textbf{Chapter \ref{chap:and-secu}} presents the current security mechanisms used in the Android system to avoid abuses from the applications and how the current security weaknesses are exploited by malicious developers.
\textbf{Chapter \ref{chap:droidwatcher}} is a presentation of the developed software \emph{DroidWatcher} as an implementation of the concepts introduced in the previous two chapters.\\

\textbf{Part \ref{part:camera}} is about wireless cameras and is composed of two chapters.\\

\textbf{Chapter \ref{chap:wifi-cam}} is an introduction to wireless cameras.
It presents the different types of cameras and the risks of discovery and compromising of a web digital camera.
\textbf{Chapter \ref{chap:cam-dcs}} is a full security analysis of the chosen camera model D-Link DCS-2130.\\

%The appendix contains documents related to mentioned as well as every software developed in the context of this thesis.
The appendix provides some reference documents used during the thesis work.
Also the appendix provides the documentation related to software developed in the scope of this thesis.

\section{Licence}

The author of this work believes in the importance of free licences for the development of technology and sharing of knowledge.
Accordingly, the content of this thesis is released under different open licences.\\

The text of the current report is licensed under a Creative Commons Attribution-ShareAlike 3.0 Unported License.
This license allows you to copy, distribute, transmit, adapt the work and to make commercial use of the work under the following conditions:

\begin{itemize}
\item You must attribute the work in the manner specified by the author or licensor (but not in any way that suggests that they endorse you or your use of the work).
\item  If you alter, transform, or build upon this work, you may distribute the resulting work only under the same or similar license to this one. 
\end{itemize}

Details and the full licence text are available at \url{https://creativecommons.org/licenses/by-sa/3.0/}.\\

The code developed and published in the context of this thesis is, with exception of explicit mention of otherwise, licensed under the GNU GENERAL PUBLIC LICENSE 3.0 licence. The full licence text is available at \url{https://www.gnu.org/licenses/gpl-3.0.txt}

%\section{Acknowledgements}



% \section{idées}

% \begin{itemize}
% \item Aujourd'hui technologie partout
% \item L'impact sur la vie privée est souvent négligé
% \item A décidé de se concentrer sur deux technologies
% \item Le smartphone
%   \begin{itemize}
%   \item Android, très à la mode auj
%   \item Nombre d'appareils en pleine croissance
%   \item Nombre de virus en croissance également
%   \item Contient de plus en plus de données personnelles
%   \item Qu'en est-il de la localisation d'un appareil ?
%   \end{itemize}
% \item La caméra de surveillance
%   \begin{itemize}
%   \item Prévue pour nous protéger (argument souvent mis en avant)
%   \item Apparition de petites caméras personnelles wifi abordables
%   \item Est-ce que ces caméras sont vraiment sécurisées ?
%   \item Analyse en détail d'un modèle en particulier
%   \item Si les faiblesses trouvées sont parfois propres à ce modèle en particulier, représentatif de l'effort mis en avant pour sécuriser ces appareils
%   \end{itemize}
% \item Document structure
%   \begin{itemize}
%   \item State of the art
%   \item android
%   \item camera
%   \item for the future
%   \end{itemize}
% %\item Thème de réflexion sur la problématique de la vie privée
% \end{itemize}