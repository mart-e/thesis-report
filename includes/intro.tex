\chapter{Thesis introduction}

\section{The context}

The usage of technology is essential in our everyday life.
It keeps us organised and manages our communications in a way we do not even realise it anymore.
This is even more true with the increasing usage of embedded devices more and more powerful.\\

This thesis does not focus on computers or websites on the Internet but on embedded devices used in our everyday life.
These devices work as a black-box, if they \emph{just work} is usually enough for the usage we have of it.
It is very convenient but this convenience often has a price, one is privacy.\\

The privacy aspect of an embedded device is usually ignored, considered as \emph{good enough}.
The confidence in the manufacturer's importance to protect is also a reccurent problem in the privacy scandals.\\


\section{Case study choices}

As part of this thesis, it has been decided to focus on two types of devices: smartphones (running Android) and wireless cameras.
This choice is justified as the privacy importance of these two devices.\\

The smartphones penetration rate is in constant raise and, in a few years, it is expected that the large majority of persons own such device in our countries.
Similarly to our personal computers, these devices contains more and more personal data.
The consequences of compromising these data can be very high for an end-user.\\

The wireless camera market is not as developed as for the smartphones but the privacy threat is as important.
The consequences of camera compromising and unauthorised access to the video stream is potentially very high and can have the opposite effect than the one expected (to facilitate abuses instead of protecting).\\

For each topic, a particular case study, the Android system and the D-Link DCS-2130 camera, has been selected.
Both case studies choices are considered as relevant and representative for the general device type.
These two choices are justified in the related parts.


\section{Thesis organisation}

The thesis is divided into two distinct parts.
The content of each part is independent and the order of reading does not affect the general comprehension.\\

\textbf{Part \ref{part:android}} is about the Android system and is composed of four chapters.\\

\textbf{Chapter \ref{chap:andro-intro}} presents and justifies the choice of studying the Android system as well as the focus on the localisation.
\textbf{Chapter \ref{chap:and-loc}} explains the localisation techniques used in Android as well as the privacy aspect related to it.
\textbf{Chapter \ref{chap:and-secu}} presents the current security mechanisms used in the Android system to avoid abuses from the applications and how the current security weaknesses are exploited by malicious developers.
\textbf{Chapter \ref{chap:droidwatcher}} is a presentation of the developed software \emph{DroidWatcher} as an application of the concepts introduced in the two previous chapters.\\

\textbf{Part \ref{part:camera}} is about wireless cameras and is composed of two chapters.\\

\textbf{Chapter \ref{chap:wifi-cam}} is an introduction to wireless cameras.
It presents the different types of cameras and the risks of discovery and compromising of a web digital camera.
\textbf{Chapter \ref{chap:cam-dcs}} is a full security analysis of the chosen camera model D-Link DCS-2130.\\

The appendix contains documents related to mentioned as well as every software developed in the context of this thesis.

\section{Licence}

The author of this work believes in the importance of free licences for the development of technology and sharing of knowledge.
Accordingly, the content of this thesis is released under different open licences.\\

The text of the current report is licensed under a Creative Commons Attribution-ShareAlike 3.0 Unported License.
This license allows you to copy, distribute, transmit, adapt the work and to make commercial use of the work under the following conditions:

\begin{itemize}
\item You must attribute the work in the manner specified by the author or licensor (but not in any way that suggests that they endorse you or your use of the work).
\item  If you alter, transform, or build upon this work, you may distribute the resulting work only under the same or similar license to this one. 
\end{itemize}

Details and the full licence text are available at \url{https://creativecommons.org/licenses/by-sa/3.0/}.\\

The media used in the report such as graphs, screenshots of interfaces and snapshots of video cameras have been realised, with the exception of explicit source mention in the related text, by the author of this thesis.
These media are also licensed under the Creative Commons Attribution-ShareAlike 3.0 Unported License.\\

The code developed and published in the context of this thesis is, with exception of explicit mention of otherwise, licensed under the GNU GENERAL PUBLIC LICENSE 3.0 licence. The full licence text is available at \url{https://www.gnu.org/licenses/gpl-3.0.txt}

\section{Acknowledgements}



% \section{idées}

% \begin{itemize}
% \item Aujourd'hui technologie partout
% \item L'impact sur la vie privée est souvent négligé
% \item A décidé de se concentrer sur deux technologies
% \item Le smartphone
%   \begin{itemize}
%   \item Android, très à la mode auj
%   \item Nombre d'appareils en pleine croissance
%   \item Nombre de virus en croissance également
%   \item Contient de plus en plus de données personnelles
%   \item Qu'en est-il de la localisation d'un appareil ?
%   \end{itemize}
% \item La caméra de surveillance
%   \begin{itemize}
%   \item Prévue pour nous protéger (argument souvent mis en avant)
%   \item Apparition de petites caméras personnelles wifi abordables
%   \item Est-ce que ces caméras sont vraiment sécurisées ?
%   \item Analyse en détail d'un modèle en particulier
%   \item Si les faiblesses trouvées sont parfois propres à ce modèle en particulier, représentatif de l'effort mis en avant pour sécuriser ces appareils
%   \end{itemize}
% \item Document structure
%   \begin{itemize}
%   \item State of the art
%   \item android
%   \item camera
%   \item for the future
%   \end{itemize}
% %\item Thème de réflexion sur la problématique de la vie privée
% \end{itemize}