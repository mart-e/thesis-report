\chapter{Thesis conclusion}
%\addcontentsline{toc}{chapter}{Thesis conclusion}

Through this thesis, several points have been raised.
The focus made on the two case studies are believed as representative of the current privacy problematic.

\section{Android security has to evolve}

The Android system is not technically unsecured with the exception of some unavoidable security flaws.
However the security issues are present and explained by two factors.
By the negligence or ignorance of the security risks during the installation process, users will install malicious applications that could have been avoided.
Applications similar to DroidWtacher are technically possible but the risk of propagation could be easily minimised with a better security awareness.

However, the user only is not responsible for security issues.
In comparison to the model used in iOS, the open politic of Android is also responsible for the higher risk.
An open politic has indubitable advantages for the user freedom but enables abuses and reduce the action of Google (left to manufacturer responsibility).
The permission system is also not perfect and enables too easily abusive behaviours.\\

It is questionable if Android will be able to react to the recent strong raise of malicious applications.
Will antivirus softwares be mandatory as on the Windows platform or will a closed and certified marketplace such as the App Store be developed?
It is however imaginable that an evolution in both the user behaviour and the platform security measures will happen to adapt to this risk.
The fragmentation of the market should also be resolved by a better reaction to security threads to manufacturer (better support of older models...).

\section{Wireless camera must come secured}

The analysis of the DCS-2130 reveals several security issues that can be exploited.
The usage of such camera without taking additional security measures is  considered as unsecured and unadvised.
It is possible to use the camera in a secure way but it requires a knowledge of security principles that most of the users do not have.\\

These security issues are not impossible to fix but it requires the manufacturer to dedicate more importance into security.
The security of appliances such as a wireless camera need be present \emph{out-of-the-box}, not requiring additional effort from the end-user.

\section{Privacy does matter}

On both Android systems and the wireless cameras, the potential leak of information has very dangerous consequences.
A smartphone is able to track easily every movement of a user or reveal sensible private information.
The case of camera abuses is easily imaginable as harmful as the burglar scenario showed.\\

The privacy is an important aspect of the usage of technology.
If the state of care for privacy does not improve, scandals such as the one related to the cache database in iPhone and Android systems or the TRENDnet hack will happen again.

\section{Beware of convenience}

The common aspects of these technologies is the convenience that drives their adoption.
As these technologies are used as a service, they are considered only for the task they achieve and not as the system they are.
The tendency of today is to abstract all the complexity from the user and sell a device that \emph{just works}.
However this is often untrue, a system is more complex than it is advertised.
Security and privacy are some of the limitations of this abstraction.

% Considering a device as a function brings risky behaviours in term of management of this device.
% It is very uncommon to apply specific security measures or to update such devices.
% Manufacturers do not necessarily provide updates of their devices as it is often seen in Android devices.
% If it is not securely configured by default, it will probably stay that way during all the device usage.

\section{Personal thoughts}

I found the researches and analysis made during this thesis very interesting.
The redaction of this document helped me to developed a critical view of the technologies we use everyday.
The analysed case studies have all security issues and enable potential abuses in term of privacy that I did not suspected.
I hope that reading this thesis will create a better awareness in the fact these technologies are not perfect and that relying fully on them is not without risk.
Fixing the security and privacy issues is not possible in one day but starts with a better concern in these issues.