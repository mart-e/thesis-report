\chapter{Thesis conclusion}

Through this thesis, several points have been raised.
The focus made on the two case studies are believed as representative of the current privacy problematic nowadays.

\section{Android}

The Android system is not technically unsecured.
With the exception of some unavoidable security flaws, it is the usage of the end users that causes the security issues.
The possibilities of an application to abuse the resources of a smartphone exists and are permitted by a negligence during the installation process.
Applications similar to DroidWtacher are possible.\\

However, the user only is not responsible for security issues.
In comparison to the model used in iOS, the open politic of Android is also responsible for the higher risk.
The permission system is also not perfect and allows too easily abusive behaviours.
It is questionable if Android will be able to react to the very quick raise of malicious applications lately.
Will antivirus softwares be mandatory as on the Windows platform or will a closed and certified marketplace such as the App Store be developed ?

\section{Wireless camera}

The analysis of the DCS-2130 reveals several security issues that can be exploited.
The usage of such camera without taking additional security measures is  considered as unsecured and unadvised.
It is possible to use the camera in a secure way but it requires a knowledge of security principles that most of the users do not have.

\section{Privacy does matter}

On both Android systems and the wireless cameras, the potential leak of information has very dangerous consequences.
A smartphone is able to track easily every movement of a user or reveal sensible privates information.
The case of camera abuses is easily imaginable as harmful as the burglar scenario showed.\\

The privacy is an important aspect of the usage of technology.
If the state of care for privacy is not improved, similar scandals such as the one related to the cache database in iPhone and Android systems or the TRENDnet hack will happen again.

\section{Convenience of appliance}

The common aspects of these technologies is the convenience that guides their usage.
As these technologies are used as a service, they are considered only for the task they achieve and not as the system they are.
The tendency of today is to hide all the complexity to the user and sell a device that \emph{just work}.
However this is often untrue, a system is more complex than it is advertised and has the limitations of this abstraction appears in term of security and privacy.\\

Considering a device as a function brings risky behaviours in term of management of this device.
It is very uncommon to apply specific security measures or to update such devices.
Manufacturers do not necessarily provide updates of their devices as it is often seen in Android devices.
If it is not securely configured by default, it will probably stay that way during all the device usage.


% \begin{itemize}
% \item Android
%   \begin{itemize}
%   \item Future ?
%   \item Ouverture -> risque sécurité
%   \item Parallele Windows
%   \end{itemize}
% \item Camera
%   \begin{itemize}
%   \item Pas sécurisé de base
%   \item Les gens ne vont pas prendre des mesures
%   \item Les gens ne vont pas mettre à jour
%   \end{itemize}
% \end{itemize}