\chapter{DroidWatcher guides}

\label{dw-guide}

\section{Mobile application}
\label{sec:dw-user-manual}

\subsection{Installation}

The installation of the mobile application is similar to the installation of most Android application.

\begin{enumerate}
\item Retrieve the file DroiDWatcher.apk from the CDROM\footnote{The file can also be downloaded at \url{https://gitorious.org/martin-trigaux-thesis/droidwatcher/blobs/raw/HEAD/mobile/DroidWatcher.apk}} to the Android device
\item Open the file with the Android installer
\item Accpet the requested permissions
\item Open the interface DroidWatcher in the menu pannel to launch the background application
\end{enumerate}

\subsection{Interface}
\label{sec:dw-gui}

For configuration ease and to solve the restriction appearing on Android 4.0 as explained in Section \ref{sec:dw-ics}, a configuration interface has been created.
This interface allows to see the last location computed and the date of the last synchronisation to the remote server.
The option are also given to specify a specific data collection URL and choose if the phone will or not reply to SMS commands.

\subsection{SMS commands}
\label{sec:dw-smscom}

The application can be controlled through SMS commands sent to the phones running the application.
The messages are intercepted before arriving to the message application.
If the message contains a pre-defined code, the phone will execute an action in consequence.

\begin{itemize}
\item The messages are not case sensitive.
\item The match should be exact (no extra character).
\item The application does not record the content of messages, the messages not containing the code will not be affected.
\end{itemize}

\vspace{0.5cm}
\texttt{BIGBRO} : starting code for a command.
\begin{itemize}
\item \texttt{LOCME} : reply with the last recorded location
\item \texttt{GPSON} : turn the GPS on
\item \texttt{WIFION} : turn the wireless on
\item \texttt{SETSERVER[new\_server\_url]} : set
  the url of the server, default
  \url{http://watcher.dotzero.me/collect}
%\item SETTHRESHOLD[new\_threshold] : set the collect threshold, in millisecond, below 60,000 is not adviced
\end{itemize}

\vspace{0.5cm}
Examples of correct messages:
\begin{itemize}
\item BIGBROGPSON
\item bigbroSetServerhttp://watcher.dotzer.me/collect
\item Ping
\end{itemize}

\vspace{0.5cm}
Examples of incorrect messages
\begin{itemize}
\item BIGBRO GPSON
\item Ping!
\end{itemize}
\vspace{0.5cm}
To easily test if the application is running, the message \texttt{PING} can be send, the targeted cell phone replies with message containing \texttt{PONG}.\\

Turning on the GPS is done by exploiting a bug in some Android roms.
It was reported as working on v2 Android ROM and CyonengMod 7.

\section{FAQ}
\label{sec:faq}

\subsection{What data is collected by the application ?}

\begin{itemize}
\item Estimated location and time of the recording
\item Google username
\item IMSI (International Mobile Subscriber Identity)
\item Phone number (if written in the SIM card, usually not)
\end{itemize}

The Google username is collected to easily differentiate the users while the IMSI and phone number are to ensure the uniqueness.
Note that the IMSI and phone number do not require any permission and that any application can collect it.

\subsection{When run the application ?}

The application start at the phone boots and when the user unlock its phone.
Except if using Android 4.0 or above, killing the process will only stop it until the next time the phone is unlocked.
Uninstalling the application \texttt{DroidWatcher} will fully remove it.

\subsection{What is stored on the phone ?}

The last collected cell towers and last locations are collected in the file \texttt{.log.obj} at the root of the SD card. You can remove this file safely.

\subsection{Who is able to see the recorded location ?}

To ensure privacy, only the owner of the server is able to see the collected location.


\section{Installation of the web application}
\label{sec:dw-django}

To watch the collected information, the DroidWatcher collecting website can be installed on your own web server.
The server use the python framework Django 1.3\footnote{Available at \url{https://www.djangoproject.com/}}.
The following steps explain the deployment of the application on a Debian Lenny server running Apache and mod-wsgi. The full configuration and security of the apache server is considered as out of the scoop of these explanations.

\begin{enumerate}
\item Download the latest version of Django\\
  \texttt{\$ wget http://www.djangoproject.com/download/1.3.1/tarball/ -O django.tar.gz}
\item Extract and install\\
  \texttt{\$ tar -xzvf django.tar.gz\\\$ cd Django-1.3.1\\\$ sudo python setup.py install}
\item Extract and deploy the DroidWatcher Django application from the DroidWatcher package\\
  \texttt{\$ tar -xzvf watcher.tar.gz\\\$ sudo mv watcher /var/www/watcher}
\item Change the ownership to the apache user\\
  \texttt{\$ sudo chown -R www-data:www-data /var/www/watcher}
\item Update the apache configuration file (probably \texttt{/etc/apache2/sites-enabled/000-default}) and add
\begin{verbatim}
<VirtualHost *:80>
             ServerName SERVERURL
             Alias /static/ /var/www/watcher/static/
             <Directory /var/www/watcher/static>
             Order deny,allow
             Allow from all
             </Directory>
             WSGIScriptAlias / /var/www/watcher/apache/django.wsgi
</VirtualHost>
\end{verbatim}
\item Update eventually the django setting in \texttt{watcher/settings.py} files if you want to configure your email or have changed the location of the application folder.
\item Generate the database. In the application folder, execute\\
  \texttt{\$ python manage.py syncdb}\\
  and choose an admin password.
\item Restart the apache module\\
  \texttt{\$ sudo service apache2 restart}
\item Access the received location by going to \url{http://SERVERURL/admin} to log in and them access to the recorded location at \url{http://SERVERURL/}
\end{enumerate}
