
\chapter{Software}
\label{chap:app-soft}

For the realisation of this thesis, several software have been developed and used for the researches.
This appendix gives a short summary of the usage and purpose of the software.\\

The code of every software mentioned here can be found on \url{https://gitorious.org/martin-trigaux-thesis/code/}.


\section{androdump.py}

Download the content of the file system on the Android device.
The ADB utility should be installed and the device connected to the computer making the dump.
A root access is required to be able to retrieve the full content of the system.
This program is used in experiments in Section \ref{sec:andro-perso-research} to monitor impact of actions on the file system.

\begin{verbatim}
Python 3
Usage: androdump.py [options]

Options:
  -v, --verbose         Enable verbose mode, lots of text
  -p PATH, --path=PATH  Starting path, default '/'
  -o OUT, --out=OUT     Outgoing path, default current path
\end{verbatim}

\section{compare\_dump.py}

Compare the content of two folders.
The hash function \texttt{sha256} is used to compare two files and detect the changes.
The script list the files that have changed were added or deleted between the two folders.
This program is used in experiments in Section \ref{sec:andro-perso-research} to compare two dumps of the system made with \texttt{androdump.py}.

\begin{verbatim}
Python 3
Usage: compare_dump.py [options]

Options:
  -v, --verbose  Enable verbose mode, lots of text
  --path1=PATH1  Starting path for folder n°1
  --path2=PATH2  Starting path for folder n°2
\end{verbatim}

\section{LocateMe.apk}

LocateMe is an Android application.
At the launch of the application, it will display current wireless information such as the number of surround wireless access points and GSM cell towers.
It will also make a single localisation request using the wireless network.
This application was used for experiments in Section \ref{sec:andro-perso-research} as a simple way to create a localisation request and observe the according changes.


\section{gen\_https.py}

Generate a specific number of HTTPS connections to a specified host.
This program is used to generate a large number of bytes of entropy to test the quality of the PRNG as explained in Section \ref{sec:dcs-random}.

\begin{verbatim}
Python 3 
usage: Generate HTTPS connexions [-h] [--host HOST] N

positional arguments:
  N            the number of connexions to generate

optional arguments:
  --host HOST  IP address of the host to contact
\end{verbatim}

\section{pcap\_to\_random.py}

Retrieve the 28 random bytes from the \emph{server hello} packet during the initialisation of an HTTPS connection.
It uses a PCAP trace file and outputs an ASCII sequence of bits.
This program is used to extract a large quantity of random bits from a trace file to test the quality of the PRNG as explained in Section \ref{sec:dcs-random}.

\begin{verbatim}
Python 2
usage: pcap_to_random.py [-h] [-s SOURCE] [-d] filename

A pcap random number extractor

positional arguments:
  filename              The pcap file to parse

optional arguments:
  -s SOURCE, --source SOURCE
                        The packet author address in the form
                        00:11:22:33:44:55
\end{verbatim}

\section{dcs-detection.py}

This program scans a range of IP to detect DLink DCS-2130 cameras available on that range.
The use of this script is explained in Section \ref{sec:dcs-web-access}.

\begin{verbatim}
Python 3
usage: dcs-detection.py [-h] [-v] range

Scan a range of IP address to detect DCS-2130 cameras

positional arguments:
  range          The range to scan in the form 1.2.3.x

optional arguments:
  -v, --verbose  Verbose mode
\end{verbatim}

\section{shodan-dcs.py}

This program uses the Shodan search engine presented in Section \ref{sec:shodanhq} to retrieve DLink DCS-2130 cameras indexed.
An API key should be created at \url{http://shodanhq.com} and used for the variable \texttt{SHODAN\_API\_KEY}.
The use of this script is explained in Section \ref{sec:dcs-web-access}.

\begin{verbatim}
Python 3
usage: shodan-dcs.py
\end{verbatim}