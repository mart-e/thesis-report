
\chapter{Softwares}
\label{chap:app-soft}

For the realisation of this thesis, several softwares have been developed and used for the researches.
This appendix gives a short summuray of the usage and purpose of the software.\\

The code of every software mentioned here can be found on \url{https://gitorious.org/martin-trigaux-thesis/code/}.

\section{gen\_https.py}

Generate a specific number of HTTPS connections to a specified host.
This program is used to generate a large number of bytes of entropy to test the quality of the PRNG as explained in Section \ref{sec:dcs-random}.

\begin{verbatim}
Python 3 
usage: Generate HTTPS connexions [-h] [--host HOST] N

positional arguments:
  N            the number of connexions to generate

optional arguments:
  --host HOST  IP address of the host to contact
\end{verbatim}

\section{pcap\_to\_random.py}

Retrieve the 28 random bytes from the \emph{server hello} packet during the initialisation of an HTTPS connection.
It uses a PCAP trace file and outputs an ASCII sequence of bits.
This program is used to extract a large quantity of random bits from a trace file to test the quality of the PRNG as explained in Section \ref{sec:dcs-random}.

\begin{verbatim}
Python 2
usage: pcap_to_random.py [-h] [-s SOURCE] [-d] filename

A pcap random number extractor

positional arguments:
  filename              The pcap file to parse

optional arguments:
  -s SOURCE, --source SOURCE
                        The packet author address in the form
                        00:11:22:33:44:55
\end{verbatim}